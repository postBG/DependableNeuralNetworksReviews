\documentclass[10pt,twocolumn,letterpaper]{article}

\usepackage{cvpr}
\usepackage{times}
\usepackage{epsfig}
\usepackage{graphicx}
\usepackage{amsmath}
\usepackage{amssymb}

% Include other packages here, before hyperref.

% If you comment hyperref and then uncomment it, you should delete
% egpaper.aux before re-running latex.  (Or just hit 'q' on the first latex
% run, let it finish, and you should be clear).
\usepackage[breaklinks=true,bookmarks=false]{hyperref}

\cvprfinalcopy % *** Uncomment this line for the final submission

\def\cvprPaperID{****} % *** Enter the CVPR Paper ID here
\def\httilde{\mbox{\tt\raisebox{-.5ex}{\symbol{126}}}}
\def\etal{\textit{et al.}}

% Pages are numbered in submission mode, and unnumbered in camera-ready
%\ifcvprfinal\pagestyle{empty}\fi
\setcounter{page}{4321}
\begin{document}

%%%%%%%%% TITLE
\title{Class-wise Feature Distribution Matching for Domain Generalization}

\author{Seungmin Lee\\
Seoul National University\\
{\tt\small profile2697@gmail.com}
% For a paper whose authors are all at the same institution,
% omit the following lines up until the closing ``}''.
% Additional authors and addresses can be added with ``\and'',
% just like the second author.
% To save space, use either the email address or home page, not both
}

\maketitle
%\thispagestyle{empty}

%%%%%%%%% ABSTRACT
\begin{abstract}
 Domain generalization (\textit{DG}) aims to learn a model that generalizes well to an unseen domain (\textit{target domain}), which has a different distribution than known domains (\textit{source domain}). Many of the previous works try to learn domain-invariant features. These methods adopt a loss that tries to match the whole distributions of the source domains. However, these works are sub-optimal because they rarely utilize task-specific information, such as class labels. For concerning the information, we propose a simple but effective regularizing method. The proposed method attempts to match the feature distributions of the same classes. By doing this, a model is expected to learn more task-specific and robust features than the previous works.
\end{abstract}

%%%%%%%%% BODY TEXT
\section{Introduction}
Deep learning has been remarkably successful in many areas~\cite{}. However, many studies find out deep learning methods are hard to generalize when they encounter an unseen domain, which has a different data distribution than domains used for training~\cite{}. This problem called \textit{domain shift}~\cite{}. For alleviating the domain shift problem, some studies have been carried out on different assumptions. Domain Adaptation (DA) assumes there are two domains~\cite{}. The first is a fully-labeled source domain, and the other is a sparsely labeled or totally unlabeled target domain. Otherwise, domain generalization (DG) assumes there are some fully-labeled source domains, but the target domain is totally unavailable. DG is a challenge but an important research area because generalization to other domains is crucial to make a safe artificial intelligence. Moreover, it is helpful to understand how deep neural networks see the world.



\section{Related Works}
\subsection{Multi-Domain Learning and Multi-Source Domain Adaptation}
The primary purpose of Multi-Domain Learning (\textit{MDL}) is to learn a single model that can compactly represent all domains with a smaller number of parameters~\cite{yang2015mdlmtl, Rebuffi17, Bilen17, Rebuffi18}. For this purpose, Bilen~\etal~\cite{Bilen17} adopts shared model parameters except for batch normalization parameters and instance normalization parameters. Rebuffi~\etal~\cite{Rebuffi17} transforms the standard residual network architecture to share a significant amount of parameters between different domains.

Multi-Source Domain Adaptation (\textit{MSDA}) also uses a set of domains, but it additionally utilizes the images of an unlabeled target domain. The main focus of MSDA is to train a model that works well on the target domain without labels of it. Even though many studies have been conducted on single-source domain adaptation, there are a limited number of researches on MSDA~\cite{Zhao2018NIPS, Chang2019cvpr, guo2018-multi, peng2018moment}. Chang~\etal~\cite{Chang2019cvpr} proposes a domain-specific batch normalization with shared weights parameters and extends their method to MSDA. Peng~\etal~\cite{peng2018moment} suggests reducing moment distances between different domains. The moment distance measures the difference between feature distributions of two domains without concerning the task at hand.

MDL and MSDA are closely related to DG since DG also utilizes many sources in many cases. However, DG is different from MDL in that the primary focus of the DG is to learn semantic and domain-invariant features, not to learn compact representations. Moreover, DG is more challenging than MSDA because DG can not utilize the target domain in training.


\subsection{Domain Generalization}
Even though existing DG methods basically aim to learn domain-invariant features, these can be classified into several groups based on their strategies. The first group proposes a novel architecture~\cite{Khosla12undobias, Li2017dg}. The methods basically separate domain-specific parameters and domain-agnostic parameters. After that, they only extract and utilize the domain-agnostic parameters for the unseen domain. The second group of methods suggests optimization algorithms that adopt meta-learning or episodic learning~\cite{li2019episodic, Li2018MLDG, NIPS2018_metareg}. For example, MLDG~\cite{Li2018MLDG} constructs an episode by splitting the source domains into training domains and test domains in each iteration. The final group of methods uses losses that aims to learn domain-invariant features~\cite{Ghifary2015mtae, muandet2013domaingeneralization, mmdaaecvpr2018, carlucci2019domain}. The losses such as maximum mean distribution (\textit{MMD}) just try to match the feature distributions of all available source domains without concerning the task at hand. Therefore, methods that adopted MMD show sub-optimal performances~\cite{Saito2018, Saito2018b}. We propose a regularizing method using a simple consistency loss that explicitly utilizes the task-specific information. By using this simple loss, we expect that the model can learn domain-invariant but semantic features.



{\small
\bibliographystyle{ieee_fullname}
\bibliography{egbib}
}

\end{document}
