\documentclass[10pt,twocolumn,letterpaper]{article}

\usepackage{cvpr}
\usepackage{times}
\usepackage{epsfig}
\usepackage{graphicx}
\usepackage{amsmath}
\usepackage{amssymb}

\usepackage[ruled]{algorithm}
% \usepackage[options]{algorithm2e}
\usepackage{algpseudocode}
\usepackage{multirow}
\usepackage{booktabs}
\usepackage{subcaption}

% Include other packages here, before hyperref.

% If you comment hyperref and then uncomment it, you should delete
% egpaper.aux before re-running latex.  (Or just hit 'q' on the first latex
% run, let it finish, and you should be clear).
\usepackage[breaklinks=true,bookmarks=false]{hyperref}

\cvprfinalcopy % *** Uncomment this line for the final submission

\newcommand{\D}{\mathcal{D}}
\def\cvprPaperID{****} % *** Enter the CVPR Paper ID here
\def\httilde{\mbox{\tt\raisebox{-.5ex}{\symbol{126}}}}
\def\etal{\textit{et al.}}

% Pages are numbered in submission mode, and unnumbered in camera-ready
%\ifcvprfinal\pagestyle{empty}\fi
\setcounter{page}{1}
\begin{document}

%%%%%%%%% TITLE
\title{Class-wise Feature Distribution Matching Regularization\\ for Domain Generalization}

\author{Seungmin Lee\\
Seoul National University\\
{\tt\small profile2697@gmail.com}
% For a paper whose authors are all at the same institution,
% omit the following lines up until the closing ``}''.
% Additional authors and addresses can be added with ``\and'',
% just like the second author.
% To save space, use either the email address or home page, not both
}

\maketitle
%\thispagestyle{empty}

%%%%%%%%% ABSTRACT
\begin{abstract}
 Domain generalization (\textit{DG}) aims to learn a model that generalizes well to an unseen domain (\textit{target domain}), which has a different distribution than known domains (\textit{source domains}). Many of the previous works try to learn domain-invariant features. These methods adopt a loss that tries to match the whole distributions of the source domains. However, these works are sub-optimal because they rarely utilize task-specific information, such as class labels. Concerning the information, we propose a simple regularizing method called Class-wise Feature Distribution Matching (\textit{FDM}). The proposed method induces a model to produce similar features when the labels of examples are the same, regardless of the examples' domains. By doing this, the model is expected to learn more task-specific and robust features than the previous works. To demonstrate the proposed methods, we conduct experiments on various settings. The proposed method consistently shows improvement compared to baseline. However, the analyses also reveal that domain-invariant features do not guarantee high performance.
\end{abstract}

%%%%%%%%% BODY TEXT
\section{Introduction}
Deep learning has been remarkably successful in many areas~\cite{}. However, many studies find out deep learning methods are hard to generalize when they encounter an unseen domain, which has a different data distribution than domains used for training~\cite{}. This problem called \textit{domain shift}~\cite{}. For alleviating the domain shift problem, many studies have been carried out on different assumptions. Domain Adaptation (DA) assumes there are two domains~\cite{}. The first is a fully-labeled source domain, and the other is a sparsely labeled or totally unlabeled target domain. Otherwise, domain generalization (DG) assumes there are some fully-labeled source domains, but the target domain is totally unavailable. DG is a challenge but important research area because generalization to other domains is crucial to make a safe artificial intelligence. Moreover, it is helpful to understand how deep neural networks see the world.

Existing DG studies can be classified into several categories depending on their strategies. Some methods proposed novel model architectures that are robust to domain shift~\cite{}. Others suggested learning algorithms aim to induce a model to fit in a more robust minimum~\cite{}. The others adopted losses to learn domain-invariant features by matching feature distributions of the source domains~\cite{}. Although these domain-invariant feature learning methods work well, the methods are sub-optimal because they do not utilize task-specific information such as class labels explicitly. 

Our approach also aims to learn domain-invariant features, but the proposed method explicitly adopts the task-specific information. More specifically, we add a simple consistency loss that induces the model to produce similar features when the labels are the same. By doing this, the model is expected to learn features that are adequately semantic and constant across domains.

To demonstrate the proposed method works, we will conduct experiments on the standard DG benchmarks such as PACS~\cite{} or VLCS~\cite{}. Specifically, we will show the consistency loss is helpful to improve the performance of the baseline, which simply aggregates all the source domains and uses it as a training set. Then, we will show that the proposed method is comparable to or better than many previous works.



\section{Related Works}
\subsection{Multi-Domain Learning and Multi-Source Domain Adaptation}
The primary purpose of Multi-Domain Learning (\textit{MDL}) is to learn a single model that can compactly represent all domains with a smaller number of parameters~\cite{yang2015mdlmtl, Rebuffi17, Bilen17, Rebuffi18}. For this purpose, Bilen~\etal~\cite{Bilen17} adopts domain-specific parameters of instance normalization and batch normalization while using shared parameters in other layers. Rebuffi~\etal~\cite{Rebuffi17} transforms the standard residual network architecture to share a significant amount of parameters between different domains.

Multi-Source Domain Adaptation (\textit{MSDA}) also uses a set of source domains, but it additionally utilizes the images of an unlabeled target domain. The main focus of MSDA is to train a model that works well on the target domain without labels of it. Even though many studies have been conducted on single-source domain adaptation, there are a limited number of researches on MSDA~\cite{Zhao2018NIPS, Chang2019cvpr, guo2018-multi, peng2018moment}. Chang~\etal~\cite{Chang2019cvpr} proposes a domain-specific batch normalization with shared weights parameters and extends their method to MSDA. Peng~\etal~\cite{peng2018moment} suggests a way to reduce the moment distance between different source domains as well as reducing the distance between target and source domains. The moment distance measures the difference between feature distributions of two domains without concerning the task at hand.

MDL and MSDA are closely related to DG since DG also utilizes a set of domains as training data. However, DG is different from MDL in that the primary focus of DG is to learn semantic and domain-invariant features, not to learn compact representations. Furthermore, DG is more challenging than MSDA because the target domain is totally unavailable in DG.


\subsection{Domain Generalization}
Even though existing DG methods basically aim to learn domain-invariant features, the methods can be classified into several groups based on their approaches. The first group proposes a novel architecture~\cite{Khosla12undobias, Li2017dg}. The methods separate domain-specific parameters and domain-agnostic parameters. After that, they only extract and utilize the domain-agnostic parameters for the unseen domain. The second group of methods suggests optimization algorithms that adopt episodic learning or self-supervised learning~\cite{li2019episodic, Li2018MLDG, NIPS2018_metareg, carlucci2019domain}. For example, MLDG~\cite{Li2018MLDG} constructs an episode by splitting the source domains into training domains and test domains in each iteration. The final group of methods uses losses that aims to learn domain-invariant features~\cite{Ghifary2015mtae, muandet2013domaingeneralization, mmdaaecvpr2018}. These methods often adopt maximum mean distribution (\textit{MMD}) constraints. However, MMD simply tries to match the feature distributions of all available source domains without concerning the task at hand. Therefore, methods that adopted MMD can be sub-optimal~\cite{Saito2018, Saito2018b}. Otherwise, we propose a regularizing method using a simple consistency loss that explicitly utilizes the task-specific information. By using this simple loss, we expect that the model can learn domain-invariant but semantic features.

\section{Proposed Method}
\subsection{Problem Setting and Notation}
In DG, we assume that there are $n$ source domains $\D = \{\D_1, \D_2, \dots, \D_n\}$ where $\D_i$ indicates $i$-th source domain which contains sample-label pairs $\{x_i^j, y_i^j\}$. Using the source domains, our goal is to train a model $h$ that performs well on unseen target domain $\D_t$. We assume the model $h$ consists of a feature extractor $f$ and a classifier $c$. 

\subsection{Deep-All Method}
The Deep-All Method is a simple but effective baseline. In this method, we just aggregate all examples of source domains and train the model using the aggregated samples. If we work on a classification task, we can use cross-entropy loss as follows:
\begin{equation}
\label{eq:agg}
\begin{aligned}
L_{all} = \mathbb{E}_{\D_s\sim\D} \big[ \mathbb{E}_{\mathbf{x}_i,y_i\sim \mathcal{D}_s} \big[  {\mathbf{y}_i}^{T} \log h(\mathbf{x}_i) \big] \big]
\end{aligned}
\end{equation}
where $\mathbf{y}_i$ is a one-hot vector representation of $y_i$

\subsection{Class-wise Feature Distribution Matching Regularization}
We propose a consistency loss called Class-wise Feature Distribution Matching Regularization (\textit{FDM}), which tries to make a model generate similar features when the labels are the same. The FDM loss is calculated as follows: For each iteration, the feature extractor produces features of the source domains. After that, we calculate the averages of the features by classes. Lastly, the FDM loss is calculated as a consistency loss between the averages and the features that share the same label. The FDM loss is defined as follows:
\begin{equation}
\label{eq:fdm}
\begin{aligned}
L_{fdm} = \mathbb{E}_{\D_s\sim\D} \big[ \mathbb{E}_{\mathbf{x}_i,y_i\sim \mathcal{D}_s} \big[  \mathbb{D}_{KL}(f(\mathbf{x}_i)|| \mathbf{m}_{y_i}) \big] \big]
\end{aligned}
\end{equation}
where $\mathbb{D}_{KL}(\cdot || \cdot)$ represents Kullback–Leibler divergence and $\mathbf{m}_{y_i}$ is the average feature of class $y_i$. For simplicity, $\mathbf{m}_{y_i}$ is calculated for each batch. The $L_{fdm}$ regularize the feature extractor $f$ to extract similar features when the classes are the same.

Finally, overall loss function is defined as a weighted sum of $L_{all}$ and $L_{fdm}$ as follows:
\begin{equation}
\label{eq:overall}
\begin{aligned}
L = L_{all} + \lambda L_{fdm}
\end{aligned}
\end{equation}
where $\lambda$ is a hyperparameter that controlls the magnitude between the two losses.



\section{Experiments}
To demonstrate the effectiveness of the proposed method, we conduct experiments on PACS dataset~\cite{} using AlexNet~\cite{}. We follow the experiment protocol of Li~\etal~\cite{}. We use a batch size 512, and $\lambda$ is fixed to 0.1 for all tests. The results are shown in Table~\ref{tab:pacs}. As we can see, the proposed method improves the performance of the model. However, the improvement is quite marginal and sub-optimal compared to state-of-the-art, which indicates the proposed regularization is quite strong.

\begin{table*}[t]
	\centering
	\scalebox{0.7}{
		\begin{tabular}{cc|cccccccccc|cc}
			\toprule
			Source & Target & D-MTAE~\cite{Ghifary2015mtae} & DSN~\cite{bousmalis2016domain} & CrossGrad~\cite{shankar2018generalizing} & DICA~\cite{muandet2013domaingeneralization}   &  DANN~\cite{ganin2016dann} & TF-CNN~\cite{Li2017dg}  & MetaReg~\cite{NIPS2018_metareg} & MLDG~\cite{Li2018MLDG} &  Fusion~\cite{Massimiliano2018ICIP} & Epi-FCR~\cite{li2019episodic} &  AGG & FDM \\
			\midrule
			C,P,S&A& 60.3& 61.1 & 61.0 & 64.6& 63.2 & 62.9& 63.5 & \textbf{66.2} & 64.1& 64.7 & 60.4 & 63.0 \\
			A,P,S&C& 58.7& 66.5 & 67.2 & 64.5 & 67.5 & 67.0&  69.5 & 66.9& 66.8& \textbf{72.3} & 70.2 & 73.1\\
			A,C,S&P& 91.1 & 83.3 & 87.6 & \textbf{91.8} & 88.1& 89.5&  87.4 & 88.0 & 90.2 & 86.1 & 85.3 & 87.4 \\
			A,C,P&S& 47.9& 58.6 & 55.9 & 51.1 & 57.0 & 57.5& 59.1 & 59.0 & 60.1& \textbf{65.0} & 60.9 & 62.3 \\
			\midrule
			\multicolumn{2}{c|}{Ave.}& 64.5 &67.4 &  67.9 & 68.0& 69.0& 69.2& 69.9 & 70.0 & 70.3&  \textbf{72.0}& 69.2 & 71.5\\
			\bottomrule
	\end{tabular}}
	\vspace{-0.3cm}
	\caption{\small Cross-domain object recognition results (accuracy. \%) of different methods on PACS using pretrained AlexNet. Best in bold.}
	\vspace{-0.4cm}
	% \vspace{-0.4cm}
	\label{tab:agg-alex}
\end{table*}


\section{Future Works}
We are planning to improve the proposed method further. According to the results, the proposed regularization seems too strong. We are planning to find a way to soften or stabilize the regularization. Additionally, we will conduct more experiments using other datasets like VLCS~\cite{chen2013vlcs} and other models such as ResNet~\cite{He2016resnet}.

\section{Conclusion}
We suggest a simple regularization technique called Class-wise Feature Distribution Matching Regularization. We conduct various experiments to demonstrate the efficiency of the proposed method. The proposed method consistently improves the performance of the baseline. However, further analysis reveals that the domain-invariant achieved by the regularizing technique is not critical to the performance. Moreover, through the study, we could learn the invariant in terms of distribution is less important than the invariant regarding performing the task at hand.


{\small
\bibliographystyle{ieee_fullname}
\bibliography{egbib}
}

\end{document}
