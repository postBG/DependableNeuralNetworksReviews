Deep learning has been remarkably successful in many areas~\cite{}. However, many studies find out deep learning methods are hard to generalize when they encounter an unseen domain, which has a different data distribution than domains used for training~\cite{}. This problem called \textit{domain shift}~\cite{}. For alleviating the domain shift problem, some studies have been carried out on different assumptions. Domain Adaptation (DA) assumes there are two domains~\cite{}. The first is a fully-labeled source domain, and the other is a sparsely labeled or totally unlabeled target domain. Otherwise, domain generalization (DG) assumes there are some fully-labeled source domains, but the target domain is totally unavailable. DG is a challenge but an important research area because generalization to other domains is crucial to make a safe artificial intelligence. Moreover, it is helpful to understand how deep neural networks see the world.

