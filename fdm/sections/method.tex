\subsection{Problem Setting and Notation}
In DG, we assume that there are $n$ source domains $\D = \{\D_1, \D_2, \dots, \D_n\}$ where $\D_i$ indicates $i$-th source domain which contains sample-label pairs $\{x_i^j, y_i^j\}$. Using the source domains, our goal is to train a model $h$ that performs well on unseen target domain $\D_t$. We assume the model $h$ consists of a feature extractor $f$ and a classifier $c$. 

\subsection{Deep-All Method}
The Deep-All Method is a simple but effective baseline. In this method, we just aggregate all examples of source domains and train the model using the aggregated samples. If we work on a classification task, we can use cross-entropy loss as follows:
\begin{equation}
\label{eq:agg}
\begin{aligned}
L_{all} = \mathbb{E}_{\D_s\sim\D} \big[ \mathbb{E}_{\mathbf{x}_i,y_i\sim \mathcal{D}_s} \big[  {\mathbf{y}_i}^{T} \log h(\mathbf{x}_i) \big] \big]
\end{aligned}
\end{equation}
where $\mathbf{y}_i$ is a one-hot vector representation of $y_i$. This method is easy to train, but it has shown comparable performance than other works.

\subsection{Class-wise Feature Distribution Matching Regularization}
We propose a consistency loss called Class-wise Feature Distribution Matching Regularization (\textit{FDM}), which tries to make a model generate similar features when the labels are the same. The FDM loss is calculated as follows: For each iteration, the feature extractor produces features of the source domains. After that, we calculate the averages of the features by classes. Lastly, the FDM loss is calculated as a consistency loss between the averages and the features that share the same label. The FDM loss is defined as follows:
\begin{equation}
\label{eq:fdm}
\begin{aligned}
L_{fdm} = \mathbb{E}_{\D_s\sim\D} \big[ \mathbb{E}_{\mathbf{x}_i,y_i\sim \mathcal{D}_s} \big[  d(f(\mathbf{x}_i), \mathbf{m}_{y_i}) \big] \big]
\end{aligned}
\end{equation}
where $d$ measures the difference between $\mathbf{x}_i$ and $\mathbf{m}_{y_i}$. we used the square of Euclidean distance. $\mathbf{m}_{y_i}$ is the average feature of class $y_i$. For simplicity, $\mathbf{m}_{y_i}$ is calculated for each batch. The $L_{fdm}$ regularize the feature extractor $f$ to extract similar features when the classes are the same.

Finally, overall loss function is defined as a weighted sum of $L_{all}$ and $L_{fdm}$ as follows:
\begin{equation}
\label{eq:overall}
\begin{aligned}
L = L_{all} + \lambda L_{fdm}
\end{aligned}
\end{equation}
where $\lambda$ is a hyperparameter that controlls the magnitude between the two losses. We found that using linear ramp-up on $\lambda$ is helpful to stable training. Thus, we updated $\lambda$ at epoch $t$ with the following schedule:
\begin{equation}
\label{eq:rampup}
\begin{aligned}
\lambda^{(t)} = min(1, \frac{t}{T_r}) * \lambda_{max}
\end{aligned}
\end{equation}
where $T_r$ is the ramp-up period, and $\lambda_{max}$ is the maximum weight of the $L_{fdm}$.

