\documentclass[10pt,twocolumn,letterpaper]{article}

\usepackage{dependable_dnn}
\usepackage{times}
\usepackage{epsfig}
\usepackage{graphicx}
\usepackage{amsmath}
\usepackage{amssymb}
\newcommand{\D}{\mathcal{D}}

% Include other packages here, before hyperref.

% If you comment hyperref and then uncomment it, you should delete
% egpaper.aux before re-running latex.  (Or just hit 'q' on the first latex
% run, let it finish, and you should be clear).
\usepackage[pagebackref=true,breaklinks=true,letterpaper=true,colorlinks,bookmarks=false]{hyperref}

\iccvfinalcopy % *** Uncomment this line for the final submission

\def\iccvPaperID{} % *** Enter the Paper ID here
\def\httilde{\mbox{\tt\raisebox{-.5ex}{\symbol{126}}}}

% Pages are numbered in submission mode, and unnumbered in camera-ready
\ificcvfinal\pagestyle{empty}\fi

\begin{document}

%%%%%%%%% TITLE - PLEASE UPDATE
\title{Project Proposal: \\ Domain Generalization by Matching Feature Distributions \\{\rm {\normalsize Seungmin Lee (profile2697@gmail.com; 2013-11420), Dept. of Computer Science and Engineering, Seoul National University}}} 

\maketitle
\thispagestyle{empty}

%%%%%%%%% BODY TEXT - ENTER YOUR RESPONSE BELOW

%%%%%%%%%%%%%%%%%
%%%%%%%%%%%%%%%%%
\section{Main Idea and Motivation}
\paragraph{Motivation and Main Idea}  Many researchers adopt Domain Generalization (DG) to mitigate the failure of deep neural networks in unseen environments. However, most existing methods merely concentrate on classifying independently discrete labels. Unfortunately, however, in many real-world applications such as medical diagnosis, the labels in interest often have intrinsic order~\cite{RCG}.
We hypothesize that the existing methods do not maintain or encode the ordinal relationship between labels.

Based on the hypothesis, this project proposes a novel problem setting named Ordinal Domain Generalization, which aims to encode the ordinal relationship between labels in learned features under the domain shift environment. In this project, first, I will propose the proper benchmarks along with baselines. Then, based on the experimental setting, I will also devise a novel training method.

\paragraph{Justification and Impact of the Project}
The primary purpose of this project is to develop a more realistic and practical training method by considering the domain shift and the ordinal relationship between labels. If successful, we would utilize the method in various applications, such as predicting the age of humans has unseen races or diagnosing disease severity with pictures taken by a new medical imaging device, etc. Furthermore, this study will enable other researchers to understand the characteristics and limitations of learned representations and develop safer and more reliable models.


\section{The Project Plan}
First of all, I will prepare the presentation of the proposal. Then, during and after the midterm season, I will research the existing methods and develop the benchmarks. After then, I expect to propose a novel approach and experiment with it on the testbed. Using those results and research, I will equip the intermediate presentation and reports. Then, I will refine the proposed method and prepare the final representation and report.
 

%%%%%%%%%%%%%%%%%
%%%%%%%%%%%%%%%%%


{\small
\bibliographystyle{ieee}
\bibliography{egbib}
}

\end{document}
